\documentclass[10pt]{scrartcl}
\usepackage[utf8]{inputenc}
\usepackage{amsmath,amssymb,amsthm}
\usepackage{latexsym} %za valovitu strelicu
\usepackage[croatian]{babel}
\usepackage{csquotes}
\MakeOuterQuote{"}
\usepackage{marvosym}
\usepackage[unicode]{hyperref}
\usepackage{thmtools}
\declaretheorem{teorem}
\declaretheorem[sibling=teorem]{korolar}
\declaretheorem[sibling=teorem]{propozicija}
\declaretheorem[style=definition,sibling=teorem,qed=$\checkmark$]{definicija}

\begin{document}

\title{\Huge\LaTeX}
\subtitle{Druga zadaća}
\author{Martin Matijević}
\date{Zagreb, \today}
\maketitle
\tableofcontents

\section{Reference}
Telefon~\eqref{eq:referenca} je fora! Molim Vas nemojte odmah preskočiti na kraj~\ref{sec:kraj}.

\section{Relacije}
$\not\prec$ izgleda isto kao i $\nprec$, ali $\nvDash$ je puno ružniji ovako $\not\vDash$.

\section{Valovita strelica}
Ova $\leadsto$, a ne ova $\rightsquigarrow$?

\section{Rastav riječi}
Tu ide par najdužih riječi u hrvatskom jeziku - elektrokardiografija, \textit{deprofesionalizirati}, najneindustrijaliziranija, gamaradioaktivnost, elektrostimulacija, \textit{prijestolonasljednikovica}, prenezaposlenost...
\par\noindent
Tu ide par najdužih riječi u hrvatskom jeziku - elektrokardiografija, \textit{dep\-ro\-fe\-si\-o\-na\-li\-zi\-ra\-ti}, najneindustrijaliziranija, gamaradioaktivnost, elektrostimulacija, \textit{prijes\-to\-lo\-nas\-ljed\-ni\-ko\-vi\-ca}, prenezaposlenost...
\par\noindent
Tu ide par najdužih riječi u hrvatskom jeziku - elektrokardiografija, \textit{dep\-ro\-fe\-si\-o\-na\-li\-zi\-ra\-ti}, najneindustrijaliziranija, gamaradioaktivnost, elektrostimulacija, \textit{prijestolonasljednikovica}, prenezaposlenost...
\par\noindent
Tu ide par najdužih riječi u hrvatskom jeziku - elektrokardiografija, \textit{deprofesionalizirati}, najneindustrijaliziranija, gamaradioaktivnost, elektrostimulacija, \textit{prijes\-to\-lo\-nas\-ljed\-ni\-ko\-vi\-ca}, prenezaposlenost...

\section{Definicije}
\begin{definicija}
Za formulu $F$ kažemo da je \textbf{tautologija} (ili \textbf{valjana formula}) ako za svaku interpretaciju $I$ vrijedi $I(F) = 1$. Kažemo da je \textbf{antitautologija} ako za svaku interpretaciju $I$ vrijedi $I(F) = 0$.
\end{definicija}
\begin{definicija}
Neka je $S$ skup formula te $I$ interpretacija. Pišemo $I(S) = 1$ ($I(S) = 0$) ako za svaku formulu $A \in S$ vrijedi $I(A) = 1$ ($I(A) = 0$). Ako to ne vrijedi, pišemo $I(S) \not= 1$ ($I(S) \not= 0$).
\end{definicija}

\section{Matematičke trotočke i pod-okoline}
\begin{equation}
I_n=\left[\begin{array}{cccc}
1 & 0 & \cdots & 0 \\
0 & 1 & \cdots & 0 \\
\vdots & \vdots & \ddots & \vdots \\
0 & 0 & \cdots & 1
\end{array}\right]
,\quad\quad\quad\quad\quad\quad
\begin{gathered}
\prod=\prod\prod\dotsi\prod\\
\sum=\sum\sum\dotsi\sum 
\end{gathered}
\end{equation}

\section{Kvadratna jednadžba}
\begin{equation}
x_{1,2}=\dfrac{-b\pm\sqrt{b^2-4ac}}{2a}
\end{equation}

\section{Operatori sa granicama pored/iznad/ispod}
\begin{equation}\label{eq:referenca}
\begin{gathered}
\sum\nolimits_{i=1}^{5}i^i=3413
\quad
\int\nolimits_{-\frac{\pi}{2}}^\pi{\sin x}\,\mathrm{d}x = 1
\end{gathered}\tag{\Telefon}
\end{equation}
\begin{equation}
\begin{gathered}
\sum\limits_{i=1}^{5}i^i=3413
\quad
\int\limits_{-\frac{\pi}{2}}^\pi{\sin x}\,\mathrm{d}x = 1
\end{gathered}
\end{equation}

\section{Kraj}\label{sec:kraj}
Ovo je kraj, hvala na pažnji. \Smiley

\end{document}

