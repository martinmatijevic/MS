\documentclass[12pt]{scrartcl}
\usepackage[utf8]{inputenc}
\usepackage{amsmath,amssymb,amsthm}
\usepackage{latexsym} %za valovitu strelicu
\usepackage[croatian]{babel}
\usepackage{csquotes}
\MakeOuterQuote{"}
\usepackage{marvosym}
\usepackage[unicode]{hyperref}
\usepackage{thmtools}
\usepackage{multirow}
\usepackage{multicol}
\usepackage{graphicx}
\declaretheorem{teorem}
\declaretheorem[sibling=teorem]{korolar}
\declaretheorem[sibling=teorem]{propozicija}
\declaretheorem[style=definition,sibling=teorem,qed=$\checkmark$]{definicija}
\newcommand{\BI}[1]{\textbf{\textit{#1}}}

\begin{document}

\title{\Huge\LaTeX}
\subtitle{Treća zadaća}
\author{Martin Matijević}
\date{Zagreb, \today}
\maketitle
\tableofcontents

\section{Teorem sa dokazom}
\BI{Teorem s imenom i dokazom} (ni teorem ni dokaz ne moraju imati matematičkog smisla); dokaz mora biti takav da je nužno koristiti \verb!\qedhere! da ne bi oznaka za kraj dokaza završila sama u retku.
\begin{teorem}[Prvi teorem]
Ovo je teorem.
\end{teorem}
\begin{proof}
\begin{equation*}
\int\nolimits_{-\frac{\pi}{2}}^\pi{\sin x}\,\mathrm{d}x = 1 \text. \qedhere
\end{equation*}
\end{proof}


\section{Fontovi}
Riječ na engleskom jeziku u \textsl{slanted} fontu, \emph{semantički naglašeni dio teksta}, te barem jednu \verb!\math! naredbu za promjenu matematičkog $\mathbb{FONTA}$.

\section{Matrica, tablica i slika}
Barem jednu \BI{matricu} (napisanu pomoću array ili pomoću matrix), barem jednu \BI{tablicu} s ćelijom kroz više stupaca i u kojoj horizontalna linija odvaja zaglavlje od ostatka tablice, te barem jednu skaliranu \BI{sliku} u floating okolini s natpisom (caption). [Ne zaboravite predati i sliku!].
\begin{gather*}
A=\begin{bmatrix}
2   &  3    &  1    & 2 \\
5   & a-1   &  0    & 2 \\
1   &  2    & b+1   & 2 \\
0   &  1    &  2    & c
\end{bmatrix}\qquad
\text{
\begin{tabular}{lclll}
\multicolumn{3}{c}{ivan}   & \multicolumn{2}{c}{marko} \\ \hline
1 & \multirow{3}{*}{4} & 2 & a           & b           \\
3 &                    & 5 & c           & d           \\
6 &                    & 7 & e           & f          
\end{tabular}}
\end{gather*}
\begin{figure}[h]
\caption{More u \Heart}
\begin{center}
\includegraphics[scale=0.6]{slikica}
\end{center}
\end{figure}

\section{Naredba s argumentom}
Barem jednu vlastitu \BI{naredbu} koja prima argument, korištenu na više mjesta u dokumentu (sami razmislite što koristite često).

\section{Liste}
\BI{Uređena lista A:}
\begin{enumerate}
    \item jedan
    \begin{enumerate}
        \item prvi unutra
        \item drugi unutra
    \end{enumerate}
    \item dva
    \item tri
\end{enumerate}
\BI{Definicijska lista:}
\begin{itemize}
    \item jedan
    \item dva
    \item tri
\end{itemize}
\BI{Uređena lista B:}
\begin{enumerate}
    \item jedan
    \begin{enumerate}
        \item prvi unutra
        \item drugi unutra
        \begin{enumerate}
            \item prvi UNUTRA
            \item drugi UNUTRA
        \end{enumerate}
        \item treći unutra
    \end{enumerate}
    \item dva
    \item tri
\end{enumerate}

\section{Prijedlog zadatka}
\BI{Prijedlog zadatka} za pisanu provjeru iz {\LaTeX}a. Treba imati dovoljno elemenata da se može ocijeniti s 0 do 5 bodova.

\begin{equation*}
    \begin{split}
        \phi &= \phi_A\cos\left(\frac{\sqrt{\omega_0+2\Omega}-\omega_0}{2}t\right)\cos\left(\frac{\sqrt{\omega_0+2\Omega}+\omega_0}{2}t\right)= \\
        &= \left\{ \overline{\omega}= \frac{\sqrt{\omega_0+2\Omega}+\omega_0}{2} \text{, } \Delta \omega =  \sqrt{\omega_0+2\Omega}-\omega_0 \right\}= \\
        &= \phi_A\cos\left(\frac{\Delta \omega}{2}t\right)\cos(\overline{\omega}t)
    \end{split}
\end{equation*}

Bodovanje:
\begin{itemize}
    \item 1 bod za \verb!equation*! i \verb!split!
    \item 1 bod za pravilne veličine zagrada
    \item 1 bod za poravnavanje sa \&
    \item 1 bod za \verb!\cos!, \verb!\frac! i \verb!\sqrt!
    \item 1 bod za grčka slova
\end{itemize}

\end{document}
